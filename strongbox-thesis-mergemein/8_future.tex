\chapter{Future Work} \label{sec:future}

\section{Integrate \SYSTEM{} Into F2FS}

To provide full-disk encryption, \SYSTEM{} was designed to offer a transparent
cryptographic layer at the block device level. Necessarily, no specialized file
systems nor kernel interface changes are required to make use of \SYSTEM{}. As
made evident by this research, the transparency of \SYSTEM{} is a significant
source of overhead. Specifically, it is necessary for \SYSTEM{} to maintain
expensive metadata structures within the cryptographic driver to prevent fatal
overwrites by the overlying file system. These structures include the in-memory
\emph{Merkle Tree} and disk-backed \emph{Transaction Journal}, which are checked
on every I/O operation and updated on every write operation as well as the
\emph{Keycount Store}, which is mutated during \SYSTEM{}'s rekeying procedure
after an overwrite has been detected.

For F2FS, this metadata maintenance results in slowdowns of up to $3\times$
compared to unencrypted I/O (see \figref{microbench-f2fs}). This blanket expense
is made more egregious by the fact that F2FS very rarely commits overwrites (see
\tblref{overwrites}), making this metadata truly useful in only a few instances
when F2FS has not triggered garbage collection.

Instead of providing a transparent cryptographic layer below F2FS, we can
conceivably do away with a significant chunk of \SYSTEM{}'s metadata management
responsibilities by integrating the now-redundant Transaction Journal
into F2FS's own internal metadata structure, namely the Checkpoint (CP) and
Segment Information Table (SIT) structures which already maintain validity
bitmaps for segments and their main area blocks. This would provide the same
functionality as a distinct \SYSTEM{} under F2FS, but with the performance
benefit that comes with shaving down on metadata maintenance.

We can take this a step further when we consider overwrite detection and
correction. With \SYSTEM{} integrated directly into F2FS, I/O operations can be
scheduled so as to control the location and reduce the frequency of costly
overwrites, perhaps even eliminating them altogether outside of garbage
collection.

\section{Explore the Trade-off Between Energy, Performance, and Security}

\SYSTEM{} utilizes ChaCha20 as its stream cipher of choice. ChaCha20 is not the
only eligible stream cipher (see \secref{design}), nor is it the fastest in the
ChaCha family of stream ciphers~\cite{ChaCha20}. Indeed, the twenty round
``ChaCha20'' is the \textit{slowest} of the available ChaCha implementations,
which include an eight and twelve round ``ChaCha8'' and ``ChaCha12''
respectively. We selected ChaCha20 for use with \SYSTEM{} because the ChaCha
standard considers 20 rounds, but there is to our knowledge no cryptanalytic
threat towards the ChaCha8/12 implementations~\cite{ChaCha-Cryptanalysis}. Other
interesting stream ciphers include Rabbit, Trivium, and even AES in CTR mode.

In exchange for a technically looser security guarantee depending on the
selected stream cipher, there may be a navigable trade-off space to exploit for
improved performance and/or energy/power savings under certain conditions. Such
a trade-off space could be explored using our implementation of \SYSTEM{} with
minimal API changes. \SYSTEM{}'s modular design lends itself well to the dynamic
swapping of algorithms at runtime.

\section{Investigate ChaCha20 Energy Usage}

In respect to I/O operation latency, \SYSTEM{} outperforms its counterpart
dm-crypt with AES-XTS in the majority of cases. In respect to energy and power
use per I/O operation, however, the matter becomes more interesting. Though it
was not explored in this research, we note that \SYSTEM{}'s energy use was
often erratic compared to dm-crypt and is deserving of further study.

Specifically, there are several \SYSTEM{} configurations where we see a
reduction in I/O operation latency over dm-crypt accompanied by a proportional
increase in power consumption by \SYSTEM{}. Though energy usage is outside of
the scope of this research, this presents an immediate area for further inquiry.

\balance
