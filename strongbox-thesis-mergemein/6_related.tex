\chapter{Related Work} \label{sec:related}

% They use block ciphers, we use stream ciphers
Some of the most popular cryptosystems offering a confidentiality guarantee for
data at rest employ a symmetric encryption scheme known as a Tweakable
Enciphering Scheme (TES)~\cite{STES,XEX}. There have been numerous TES-based
constructions proposed for securing data at rest~\cite{STES,CMC,HCTR},
including the well known XEX-based XTS operating mode of AES~\cite{XTS} explored
earlier in this work. Almost all TES constructions and the storage management
systems that implement them use one or more block ciphers as their primary
primitive~\cite{TES-From-Stream-Cipher,STES}.

% We look a lot like a dm-crypt replacement...
Our \SYSTEM{} implementation borrows from the design of these systems. One in
particular is \emph{dm-crypt}, a Linux framework employing a
\textit{LinuxDeviceMapper} to provide a virtual block interface for physical
block devices. Dm-crypt provides an implementation of the AES-XTS algorithm among
others and is used widely in the Linux ecosystem~\cite{DmC-Android, dmcrypt}.
The algorithms provided by dm-crypt all employ block ciphers~\cite{dmcrypt}.
Instead of a block ciphers, however, \SYSTEM{} uses a stream cipher to provide
the same confidentiality guarantee and consistent or better I/O performance.
Further unlike dm-crypt and other similar virtualization frameworks, \SYSTEM{}'s
ciphering operations do not require sector level tweaks, depending on the
implementation. With \SYSTEM{}, several physical blocks consisting of one or
more sectors are considered as discrete logical units, \ie nuggets and flakes.

% Talk about STES and how we're different (and more general purpose).
Substituting a block cipher for a stream cipher forms the core of several
contributions to the state-of-the-art~\cite{STES, TES-From-Stream-Cipher}.
Chakraborty et al. proposed STES---a stream cipher based low cost scheme for
securing stored data~\cite{STES}. STES is a novel TES which can be implemented
compactly with low overall power consumption. It combines a stream cipher and a
universal hash function via XOR and is targeting low cost FPGAs to provide
confidentiality of data on USBs and SD cards. Our \SYSTEM{}, on the other hand,
is not a TES and does not directly implement a TES. \SYSTEM{} combines a stream
cipher with nonce ``tweak'' and nugget data via XOR and is targeting any
configuration employing a well-behaved Log-Structured Filesystem at some level
(LFS) to provide confidentiality of data.

% We're supposed to be a transparent layer at the block device level, but that's
% not novel either. Talk about SBD a bit. Talk about how we're different.
Offering a transparent cryptographic layer at the block device level has been
proposed numerous times~\cite{SBD}. Production implementations include storage
management systems like dm-crypt. Specifically, Hein et al. proposed the Secure
Block Device (SBD)~\cite{SBD}---an ARM TrustZone secure world transparent
filesystem encryption layer optimized for ANDIX OS and implemented and evaluated
using the Linux Network Block Device (NBD) driver. \SYSTEM{} was also
implemented and evaluated using the NBD, but is not limited to one specific
operating system. Further unlike \SYSTEM{}, SBD was not explicitly designed for
use outside of the ARM TrustZone secure world. Contrarily, \SYSTEM{} was
designed to be used on any system that provides a subset of functionality
provided by a Trusted Platform Module (TPM) and/or Trusted Execution Environment
(TEE). Specifically, \SYSTEM{} requires the availability of a dedicated hardware
protected secure monotonic counter to prevent rollback attacks and ensure the
freshness of \SYSTEM{}. The primary design goal of \SYSTEM{} was to achieve
performance parity with the industry standard AES-XTS algorithm while taking
advantage of the speedup gained by utilizing a stream cipher in place of both
AES and the XTS operating mode.

% The integrity protection scheme used by \SYSTEM{} is not novel. ZFS, SBD, etc.
% uses it. We use it to to provide integrity protection. The Merkle Tree
% approach is well-studied and we know it works. AES-XTS doesn't do integrity
% protection though.
Achieving on-disk data integrity protection through the use of checksums has
been used by filesystems and many other storage management systems. Examples
include ZFS~\cite{ZFS} and others~\cite{SBD}. For our implementation of
\SYSTEM{}, we used the Merkle Tree library offered by SBD to manage our in-
memory checksum verification. An implementation of \SYSTEM{} need not use the
SDB SHA-256 Merkle Tree library. It was chosen for convenience.

\PUNT{The use of journaling in the design is also not novel, though it's also not
meant to serve as a replacement for the overlying filesystem's crash recovery
and journaling functionality. Like Ext4 ordered journaling mode, we employ a light or
``ordered'' journaling in that we commit our metadata first and then immediately
push data out to secondary storage. Our implementation of \SYSTEM{} implements
rekeying is implemented in as full journaling}

% TODO (maybe later?): Add one more thing where you talk about AES and say that we provide
% stronger guarantees; STES, it needs one more sentence about why securing data
% on a flash drive attached to an FPGA is different than a flash drive attached
% to a processor

% Any SOSP/OSDI papers to talk about?
